% !TEX program=xelatex

\documentclass{scrartcl}

\usepackage{polyglossia, xltxtra}
\setmainlanguage{german}
\usepackage{amsmath,mathtools, amsthm, amssymb,cases}

\usepackage{scrlayer -scrpage}
\lohead{Mansur Daschaew \\ Janina Rastetter \\ Maren Raus}
\cohead{Prävalenzabhängige Kontaktdaten}
\rohead{14.02.2022}
\pagestyle{scrheadings}


%\newtheorem{defi}{Definition}[section]
%\newtheorem{satz}[defi]{Satz}
%\newtheorem{cor}[defi]{Korollar}
%\newtheorem{bem}[defi]{Bemerkung}
%\newtheorem{folg}[defi]{Folgerung}
%\newenvironment{beweis}
%	{\begin{proof}[Beweis]}
%	{\end{proof}}

\begin{document}

\begin{center}
\huge\textbf{Modellierung eines verallgemeinerten SEIR-Modells mit prävalenzabhängigen Kontaktraten}
\end{center}

\section{SEIR-Modell}

Das SEIR-Modell wird durch das folgenden System gewöhnlicher Differentialgleichungen beschrieben:
\begin{align*}
\frac{dS}{dt} &= -\beta \frac{SI}{N} \\[10pt]
\frac{dE}{dt} &= \beta \frac{SI}{N} - \alpha E \\[10pt]
\frac{dI}{dt} &= \alpha E - \gamma I \\[10pt]
\frac{dR}{dt} &= \gamma I 
\end{align*}

- verallgemeintertes SEIR Modell (GG-SEIR) kann flexibleres Verhalten einbeziehen \\
- z.B. frühes sub-exponentielles Wachstum \\
- Modell beschrieben durch obige DGL, Variablen erklären \\

\section{Kontaktrate}
- Variation der Kontaktrate $\beta(t) $ \\
- $\beta(t) = \beta_0 [(1- \phi) f(t; \theta) + \phi]$ \\
- der kleinstmögliche Wert für $\beta$ ist dann $\phi\beta_0$ \\

- exponentielle, harmonische, und hyperbolische $\beta$- Funktionen, verschiedene Parameter [Plots] \\

\section{weitere Abschnitte}









\vspace*{\fill}
\textbf{Literatur:}  \\
P. Yan, G. Chowell: Quantitative Methods for Investigating Infectious Disease Outbreaks, 2019 \\
A. King: Ordinary differential equations in R, https://kinglab.eeb.lsa.umich.edu/480/nls/de.html, Zugriff: 03.02.2022

\end{document}